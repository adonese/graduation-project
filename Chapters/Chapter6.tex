\chapter{Conclusion and Recommendations}
\label{Chapter6}

In this study we attempted to evaluate recent GRACE/GOCE models on Sudan. Our aim of this study was never to develop a new datum for Sudan, but rather to exhaustively try different models on our study area. Interestingly, our results--even without any corrections--are comparable to that reported by \citep{ahmed_msc, godah}. 

\begin{itemize}
	\item We begun our study by collecting terrestrial data for our study area
	\item GeoidApp: A software application that is mainly developed to help researchers evaluate different degrees of the model. The whole process is fully automated.
	\item The whole process of visualization, statistical analysis, reports are all handled within GeoidApp, and stored in their corresponding directories.
\end{itemize}

\section{Recommendations}

We believe that GGMs are the way to go for gravity field measurements. The traditional techniques for measuring gravity field has reached their intrinsic limitations. GGM provide global, regular and dense datasets of high and homogeneous quality. The use of GGM will lead, eventually, to the replace of expensive and time consuming spririt leveling, with the new fast, very cheap GPS/leveling. It will also contribute to the goal of unifying the regional datums.
\\
The availability of the data is a huge problem, in terms of the large un-surveyed areas, and also the access of the data. We have contacted with GETECH to get a copy of their data about Sudan, but they did not respond (as for the time we are writing this). We highly recommend to establish new GPS/leveling networks, that has more density and well distributed among Sudan. We do not recommend to use the dataset from \citep{osman} as it shows huge error compared to all of our models. 
\\
We, the authors of GeoidApp are very welcome to the use of our application in any research. We believe that our work will help researcher around the world to experiment with many different degrees of their GGMs without the need to manually tune any thing. We also highly recommend researchers to work on ``the unified datum"--geodetic main problem--which can only be solved by the means of GGM. The cost of GPS leveling, and their speed should encourage researchers and government agencies to work even harder towards 1-cm precision model.
\\
We would like to acknowledge the authors of NumPy \cite{numpy}, SciPy \cite{scipy}, Pandas \cite{pandas}, Bokeh \cite{bokeh}, \cite{octave}, and \cite{ipython}.