\chapter{Geoid Determination from GGM}

In this chapter we will present the mathematical and physical interpretations of computing the geoid from GGMs. In the previous chapter we said that one issue of classical way of solving geodetic problems is the that the density of the body (in this case the Earth), should be known. Which is clearly impossible. Our problem is that we want to know the gravitational field in outer space without the knowing the density structure of the Earth, but with the knowledge of the potential o the boundary. This kind of problem is called ``Boundary Value Problem" or BVP. In our case, even the shape of that body e.g., the Earth must be considered unknown. Which to a special type of BVP called ``Geodetic Boundary Value Problem" or GBVP.

\section{Boundary Value Problem}

Earlier, we have introduced Poisson's equation \ref{eqn:poisson}, but without any further details about it. For the sake of convenience, we will write the equation again

\begin{equation}
\triangle W = -4 \pi G \varrho
\end{equation} 
This equation is a general case of a more familiar equation in geodesy called `Laplace's equation'. Laplace's equation is a special kind of Poisson equation, it can be derived by setting the $\varrho$ to zero outside the masses

\begin{equation}
\label{eqn:laplace}
\triangle W = 0
\end{equation}

In a compact form 

\begin{equation}
\triangle W = 
\begin{cases}
-4 \pi G \varrho & \text{inside, Poisson} \\
0 & \text{outside, Laplace}
\end{cases}
\end{equation}




\subsection{Existence and Uniqueness}

The first step after developing our BVP is to prove their \textit{existence} and \textit{uniqueness}. We basically aim to prove that our BVP has a solution, and it is unique.

\subsubsection{Uniqueness}
Assuming the BVP is not unique, and we were able to find two different solutions $W_1$ and $W_2$. And let us say that their different is called $U$. That is $W_1 - W_1 = U$. Using Green's $1^{st}$ identity we can prove the existence of our BVP


So either U, or its normal derivative is zero on the boundary and since the integrand is always positive, $\nabla U$ must be zero.
And that prove the uniqueness of our $1^{st}$ BVP. For interested readers in the development and prove for the rest of BVPs refer to \cite{lndrem}.  
\\
We have proved that the solution of BVP is unique, but we did not derive that solution yet.

\section{Solving Laplace's equation}
We will start simply by solving Laplace's equation $\triangle W = 0$ in the Cartesian case, and then we will solve the problem in the spherical case. Both solutions will lead into a series of orthogonal base functions that can be solved by 1) Fourier series in the case of Cartesian solution, and 2) spherical harmonics in the case of spherical solution. The former solution (Cartesian one) is generally used in regional application, while the latter (spherical) serves more as a global solution. Hence the use of GGM.

\subsection{Cartesian coordinates}
Our task is to solve $\triangle W(x, y, z) = 0$ for $z > 0$. We start our solution by separating the variables

\begin{equation}
\triangle W(x, y, z) = \triangle f(x) \triangle g(y) \triangle h(z) = 0
\end{equation}
remember that $\triangle f$ is just a short hand notation for $\nabla \cdot \nabla f$
\begin{displaymath}
\triangle f = \nabla \cdot \nabla f = \frac{\partial^2 f}{\partial x^2} + \frac{\partial^f}{\partial y^2} + \frac{\partial^2 f}{\partial z^2}
\end{displaymath}

Applying the chain rule gives us

\begin{equation}
f"gh + fg"h + fgh"
\end{equation}
we substitute the second partial derivative symbol by " notation, just for clarification. 
\\
Dividing by $fgh$ will lead us to
\begin{equation}
\frac{f"}{f} + \frac{g"}{g} + \frac{h"}{h} = 0
\end{equation}

