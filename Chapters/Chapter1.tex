\chapter{Introduction}

Historically Sudan has two gravimetric geoid models \cite{ahmed_similar_paper}. The first one (Geoid91) was computed by \cite{fashir} in 1991 using Geodetic reference System GRS80. The free-air co-geoid was computed from the combination of surface gravity data using a modified Stokes's kernel and GODDARD EARTH MODEL (GEM-T1)  \cite{fashir}. GEM-T1, a satellite-only model, is complete to degree and order of 36 \cite{nasa}. The second model is known as `KTH-SDG08' was computed in 2008 using optimum least-squares modification of Stoke's kernels, which is widely known as KTH method \cite{ahmed_msc}. EIGEN-GRACE02S
satellite-only model was adopted for KTHSDG08 final computation at spherical harmonic degree and order 120. Adam was the first to attempt to compute the geoid for Sudan in 1967 \cite{adam}. Due to lack of data from neighboring countries and large un-surveyed areas in Northwest and Southwest parts of Sudan, Adam found that the information was insufficient to determine the accurate geoid in Sudan and recommended to fill the gabs over there \cite{ahmed_msc}. 


\section{Objectives of the thesis}
The main objective of this thesis is to evaluate recent grace/goce models over Sudan using asterogeodetic data and GPS/Leveling data. As a result we also propose a general software framework for various geoid components computations. In particular, it can be used to compute ``geoid-height'', ``geoid-undulation'`, and ``geoid-disturbance''. To the best of our knowledge, we could not find a software for geoid computations that is
\begin{itemize}
	\item {Supported}. Most of geoid computations libraries are no longer supported, their links are dead.
	\item {Strong back-end, and simple front-end}. It is common to use a low-level programming language in the calculation of geoid componenets e.g. geoid height. Low-level languages are very fast compared to high-level languages, but they are much harder. We built a framework on top of C/C++ libraries, with a very simple interface, in particular Java and Matlab.
	\item {Customizable}. Because we know that users need to explore different models with different degree. We offer them just that. GeoidApp is designed such that the user can easily modify any parameter.
\end{itemize}

In particular our contribution is
\begin{itemize}
	\item Evaluate recent Grace/Goce money in Sudan
	\item GeoidApp: A unified software framework for geoid computations
\end{itemize}


\section{Thesis structure}
Following this introduction, the thesis is divided into five chapters

\begin{itemize}
	\item Chapter 2\\
	Glances the theory behind GGMs, and summarizes dedicated satellites missions.
	\item Chapter 3\\
	Details about the input data
	\item Chapter 4\\
	Methodology
	
	
	
\end{itemize}
